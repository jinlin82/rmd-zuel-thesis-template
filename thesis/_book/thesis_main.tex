%\documentclass[UTF8,a4paper,12pt]{znufecls}  % Latex 去掉上面的语句,加上本语句
\documentclass[doctor,twoside,chapterhead,otf]{znufethesis}
% \usepackage{gbt7714}


%======================== 根据选项设置代码处理方式 RMARKDOWN 中独有=============
\usepackage{listings}




\providecommand{\tightlist}{\setlength{\itemsep}{0pt}\setlength{\parskip}{0pt}}
\newcommand{\passthrough}[1]{#1}

%%% Change title format to be more compact
\usepackage{titling}

% Create subtitle command for use in maketitle
\newcommand{\subtitle}[1]{
  \posttitle{
    \begin{center}\large#1\end{center}
    }
}

\setlength{\droptitle}{-2em}

  \title{中南财经政法大学硕士论文Rmarkdown模板}
  \pretitle{\vspace{\droptitle}\centering\huge}
  \posttitle{\par}


  \author{Jin}
  \preauthor{\centering\large\emph}
  \postauthor{\par}

  \predate{\centering\large\emph}
  \postdate{\par}
  \date{2019-04-15}


%% ===============================================================
\begin{document}
\phantomsection %
\pdfbookmark{封面}{cover}
\IfFileExists{./misc/cover.pdf}{\includepdf[pages=-]{./misc/cover}}{
\maketitle
}

\blankpage


%% ========================== Frontmatters ========================
%% -------------- 原创声明-------------
\cleardoublepage
\phantomsection %
\pdfbookmark{声明}{creative}
\begin{creative}
\vspace{1cm}
\begin{center}
{\xiaoer \songti \bfseries 学位论文独创性声明}
\end{center}


本人所呈交的学位论文,是在导师的指导下,独立进行研究所取得的成果。除文
中已经注明引用的内容外,本论文不含任何其他个人或集体已经发表或撰写的作
品。对本文的研究做出重要贡献的个人和集体,均已在文中标明。

本声明的法律后果由本人承担。
\vspace{1cm}

\hspace{20em}论文作者(签名):     

\hspace{20em}日期:\hspace{2em}年\hspace{2em}月\hspace{2em}日
\vspace{2cm}

\begin{center}
{\xiaoer \songti \bfseries 学位论文使用授权书}
\end{center}


本论文作者完全了解学校关于保存、使用学位论文的管理办法及规定,即学校有
权保留并向国家有关部门或机构送交论文的复印件和电子版,允许论文被查阅和
借阅。本人授权中南财经政法大学可以将本学位论文的全部或部分内容编入学校
有关数据库和收录到《中国博士学位论文全文数据库》进行信息服务,也可以采
用影印、缩印或扫描等复制手段保存或汇编本学位论文。

在本文获评校级以上(含校级)优秀论文的前提下,授权学校研究生院与中国知
网签订收录协议并由作者本人享有、承担相应的权利和义务。

注:保密学位论文,在解密后适用于本授权书。

\vspace{1cm}
\hspace{20em}论文作者(签名):

\hspace{20em}日期:\hspace{2em}年\hspace{2em}月\hspace{2em}日

\end{creative}
\blankpage

%% ------------- 符号列表-------------
\cleardoublepage
\frontmatter
\pagestyle{plain}

\cleardoublepage
\phantomsection %
\pdfbookmark{\denotationname}{denotation}
\pagenumbering{arabic}
\begin{denotation}
\itemsep 0em

\item [ACD] 自回归条件持续时间模型(Autoregressive Conditional Duration)
\item [ACI] 自回归条件强度模型(Autoregressive Conditional Intensity)  
\item [AIC] 赤池信息准则(Akaike Information Criterion)
\item [BIC] 贝叶斯信息准则(Bayesian Information Criterion)
\item [VaR] 风险价值(Value At Risk)
\end{denotation}
 % 如果要生成符号列表
\ifodd\thepage
\blankpage
\fi

%% ---------------- 摘要 -------------
\cleardoublepage
\frontmatter
\begin{cabstract}
  \renewcommand{\chapterlabel}{摘\hspace{2em}要}

2008年1月9日,

本文以的参考建议。 
  
  \bigbreak

  {\bfseries 关键词}:GAS;黄金期货;波动风险;滚动预测
   
\blankpage
\end{cabstract}




\begin{eabstract}
On January 9, 2008, 

Gold futures the application of measurement, provides some reference 
opinions for primary supervisors and investors.
  \\*

\end{eabstract}

\ekeywords{GAS; gold futures; volatility risk; rolling forecast }



\ifodd\therealpage
\blankpage
\fi


%% ---------------- 目录 ------------
{

\setcounter{tocdepth}{2}
\cleardoublepage
\phantomsection %
\pdfbookmark{\contentsname}{toc}
\pagenumbering{arabic}
\tableofcontents        % 生成目录

\ifodd\thepage
\blankpage
\fi
}

\cleardoublepage
\phantomsection %
\pdfbookmark{表目录}{tabletoc}
\listoftables           % 如果要生成表目录

\ifodd\thepage
\blankpage
\fi

\cleardoublepage
\phantomsection %
\pdfbookmark{图目录}{figtoc}
\listoffigures          % 如果要生成图目录

\ifodd\thepage
\blankpage
\fi
%% ==========================================================

\mainmatter
\pagestyle{mpage}

\cleardoublepage
\pagestyle{emptypage}
\renewcommand{\chapterlabel}{导论}

\hypertarget{ux5bfcux8bba}{%
\chapter*{导论}\label{ux5bfcux8bba}}
\addcontentsline{toc}{chapter}{导论}

\hypertarget{ux7814ux7a76ux80ccux666fux548cux610fux4e49}{%
\subsection{研究背景和意义}\label{ux7814ux7a76ux80ccux666fux548cux610fux4e49}}

\hypertarget{ux7814ux7a76ux80ccux666f}{%
\subsubsection{研究背景}\label{ux7814ux7a76ux80ccux666f}}

黄金兼具商品、货币以及投资属性,在国际金融市场上是一种常用的避险工具,并逐步演变为一种社会公认的价值判断标准,其资产储备和投资增值的价值远大于纸币和其他商品,其储存量也代表了国家的实力,是稳定市场经济、抵挡通胀的重要后盾\footnote{市场投资者越担忧通货膨胀的发生,黄金市场将越受青睐。例如,2008年金价创下历史新高,商品价格的上涨与通货膨胀率成正比关系,当参与者持有的现金所产生的利息无法跟上价格的上涨速度,黄金就是避风港。},也成为大众参与者稳定资金、预防通胀的常用手段,而其他商品及货币不具有在异常时段的避险功能\citep{Baklaci2016, 王安羽2011}。

\hypertarget{ux56fdux5185ux5916ux7814ux7a76ux73b0ux72b6ux8ff0ux8bc4}{%
\subsection{国内外研究现状述评}\label{ux56fdux5185ux5916ux7814ux7a76ux73b0ux72b6ux8ff0ux8bc4}}

\hypertarget{ux9ec4ux91d1ux671fux8d27ux5e02ux573aux98ceux9669ux6d4bux5ea6ux7814ux7a76ux7efcux8ff0}{%
\subsubsection{黄金期货市场风险测度研究综述}\label{ux9ec4ux91d1ux671fux8d27ux5e02ux573aux98ceux9669ux6d4bux5ea6ux7814ux7a76ux7efcux8ff0}}

1.国外关于黄金期货市场风险测度的研究现状

Baklaci等人(2016)发现新兴市场仍然是一个尚未开发的领域,而大多数样本国家的黄金期货存在波动传递,新兴黄金期货市场的风险分散和跨市场对冲\footnote{当持有黄金的需求增加时,黄金价格将迎来飙升。所以,金价基本上与股票、债券市场的价格相反,黄金是一种很好的对冲工具。}机会非有限\citep{Baklaci2016}。

2.国内关于黄金期货市场风险测度的研究现状

由于2008年中国才推出黄金期货的市场,时间相比于比国外较晚,国内学者对此研究有一定的局限性\citep{2000Learning},所使用模型主流为传统的GARCH族模型,模型普遍较为简单,分析方向多集中于与其他金融市场的联动分析和溢出效益等\citep{Bolstad2017}。王安羽(2011)利用GARCH族模型拟合波动性,并对GARCH类对波动率的不同模拟做优劣对比。结果证实不对称的GARCH模型为最优模型,对中国投资者的市场风险度量VaR和预测具有一定的作用\citep{王安羽2011}。

\hypertarget{ux9ec4ux91d1ux671fux8d27ux5e02ux573aux53caux4ef7ux683cux6ce2ux52a8ux98ceux9669ux7814ux7a76ux57faux7840}{%
\chapter{黄金期货市场及价格波动风险研究基础}\label{ux9ec4ux91d1ux671fux8d27ux5e02ux573aux53caux4ef7ux683cux6ce2ux52a8ux98ceux9669ux7814ux7a76ux57faux7840}}

\pagestyle{mpage}

\hypertarget{ux9ec4ux91d1ux671fux8d27ux5e02ux573aux98ceux9669ux7684ux5f71ux54cdux56e0ux7d20}{%
\section{黄金期货市场风险的影响因素}\label{ux9ec4ux91d1ux671fux8d27ux5e02ux573aux98ceux9669ux7684ux5f71ux54cdux56e0ux7d20}}

黄金期货市场属于金融资本交易场所,则意味着当取得投资回报时,也存在一定的市场风险。其风险大致包含市价波动、流动性、市场信用、公开操作和法律方面。从而针对不同种类的风险,就对应着不一样的影响因素。
\[y=\beta_0 + \beta_1 x_i + u_i\]
其中

\hypertarget{ux57faux4e8e-gas-ux6a21ux578bux7684ux9ec4ux91d1ux671fux8d27ux6536ux76caux7387ux5efaux6a21}{%
\chapter{基于 GAS 模型的黄金期货收益率建模}\label{ux57faux4e8e-gas-ux6a21ux578bux7684ux9ec4ux91d1ux671fux8d27ux6536ux76caux7387ux5efaux6a21}}

\hypertarget{ux6570ux636eux6765ux6e90}{%
\section{数据来源}\label{ux6570ux636eux6765ux6e90}}

\hypertarget{ux6570ux636eux9009ux53d6ux53caux610fux4e49}{%
\subsection{数据选取及意义}\label{ux6570ux636eux9009ux53d6ux53caux610fux4e49}}

不同于国外期货的选取,以最近月份的期货合约逐日交易的收盘价为基础,因为期货合约是有生命周期的,合约在最后一个交易日将会平仓终止买卖。

\hypertarget{ux57faux4e8e-gas-ux6a21ux578bux7684ux9ec4ux91d1ux671fux8d27ux5e02ux573aux98ceux9669ux6d4bux5ea6ux7814ux7a76}{%
\chapter{基于 GAS 模型的黄金期货市场风险测度研究}\label{ux57faux4e8e-gas-ux6a21ux578bux7684ux9ec4ux91d1ux671fux8d27ux5e02ux573aux98ceux9669ux6d4bux5ea6ux7814ux7a76}}

\hypertarget{ux57faux4e8e-gas-ux6a21ux578bux7684ux9ec4ux91d1ux671fux8d27ux98ceux9669ux6d4bux5ea6}{%
\section{基于 GAS 模型的黄金期货风险测度}\label{ux57faux4e8e-gas-ux6a21ux578bux7684ux9ec4ux91d1ux671fux8d27ux98ceux9669ux6d4bux5ea6}}

\hypertarget{var-ux6edaux52a8ux9884ux6d4b}{%
\subsection{VaR 滚动预测}\label{var-ux6edaux52a8ux9884ux6d4b}}

为了对同业拆借市场的利率风险进行控制,可以借助``风险价值''(Value-at-Risk,VaR)来进行准确测度。在险价值表示\passthrough{\lstinline!\_lfun!}在某个置信水平下,由于市场的正常波动,金融资产的预计最大损失,即收益率密度曲线的一个分位点。若 \(VaR>r_t\),则说明该模型在第t天具备优良的表现,可以成功预测。

\cleardoublepage
\pagestyle{emptypage}
\renewcommand{\chapterlabel}{结论与展望}

\hypertarget{ux7ed3ux8bbaux4e0eux5c55ux671b}{%
\chapter*{结论与展望}\label{ux7ed3ux8bbaux4e0eux5c55ux671b}}
\addcontentsline{toc}{chapter}{结论与展望}

\hypertarget{ux4e00ux7ed3ux8bba}{%
\subsection*{一、结论}\label{ux4e00ux7ed3ux8bba}}
\addcontentsline{toc}{subsection}{一、结论}

波动性是金融时序最关键的特征之一,也是投资者和企业以及政府行使决策的关键影响因素之一。

\hypertarget{ux4e8cux4e0dux8db3ux4e0eux5c55ux671b}{%
\subsection*{二、不足与展望}\label{ux4e8cux4e0dux8db3ux4e0eux5c55ux671b}}
\addcontentsline{toc}{subsection}{二、不足与展望}

在研究过程中不可避免地存在缺陷和遗憾,有必要更进一步的研究和改进:

\ifodd\thepage
\blankpage
\fi

%% ========================= 参考文献 =======================
\cleardoublepage
\pagestyle{emptypage}
\renewcommand{\chapterlabel}{\bibname}
\bibliographystyle{./style/gbt7714-author-year-zuel}

\bibliography{Bibfile}



\ifodd\thepage
\blankpage
\fi

%% ========================= Backmatters =============================
\appendix
\backmatter
\cleardoublepage
\pagestyle{appendixpage}
\renewcommand{\chapterlabel}{\appendixname} % 设置附录的页眉

%%------------ 程序代码----------------
%% ======================= codes ========================
\chapter{程序代码}
\begin{spacing}{1.0}
	\lstinputlisting[caption= Python 程序代码, language=Python]{./codes/02-model.py}
  	\lstinputlisting{./codes/03-estimation.R}  % 程序代码
  	\lstinputlisting{./codes/04-aqi-des.R}
\end{spacing}





\chapter{原始数据}


\begin{longtabu} to \linewidth {>{\raggedright\arraybackslash}p{4.5cm}>{\raggedleft}X>{\raggedleft}X>{\raggedleft}X>{\raggedleft}X}
\caption{\label{tab:unnamed-chunk-1}附录数据测试表}\\
\toprule
  & sex & age & ym & child\\
\midrule
\endfirsthead
\caption[]{附录数据测试表 (续)}\\
\toprule
  & sex & age & ym & child\\
\midrule
\endhead

\bottomrule
\endfoot
\bottomrule
\endlastfoot
\rowcolor{gray!6}  1 & male & 37 & 10.000 & no\\
2 & female & 27 & 4.000 & no\\
\rowcolor{gray!6}  3 & female & 32 & 15.000 & yes\\
4 & male & 57 & 15.000 & yes\\
\rowcolor{gray!6}  5 & male & 22 & 0.750 & no\\
\addlinespace
6 & female & 32 & 1.500 & no\\
\rowcolor{gray!6}  7 & female & 22 & 0.750 & no\\
8 & male & 57 & 15.000 & yes\\
\rowcolor{gray!6}  9 & female & 32 & 15.000 & yes\\
10 & male & 22 & 1.500 & no\\
\addlinespace
\rowcolor{gray!6}  11 & male & 37 & 15.000 & yes\\
12 & male & 27 & 4.000 & yes\\
\rowcolor{gray!6}  13 & male & 47 & 15.000 & yes\\
14 & female & 22 & 1.500 & no\\
\rowcolor{gray!6}  15 & female & 27 & 4.000 & no\\
\addlinespace
16 & female & 37 & 15.000 & yes\\
\rowcolor{gray!6}  17 & female & 37 & 15.000 & yes\\
18 & female & 22 & 0.750 & no\\
\rowcolor{gray!6}  19 & female & 22 & 1.500 & no\\
20 & female & 27 & 10.000 & yes\\
\addlinespace
\rowcolor{gray!6}  21 & female & 22 & 1.500 & no\\
22 & female & 22 & 1.500 & no\\
\rowcolor{gray!6}  23 & female & 27 & 10.000 & yes\\
24 & female & 32 & 10.000 & yes\\
\rowcolor{gray!6}  25 & male & 37 & 4.000 & yes\\
\addlinespace
26 & female & 22 & 1.500 & no\\
\rowcolor{gray!6}  27 & female & 27 & 7.000 & no\\
28 & male & 42 & 15.000 & yes\\
\rowcolor{gray!6}  29 & male & 27 & 4.000 & yes\\
30 & female & 27 & 4.000 & yes\\
\addlinespace
\rowcolor{gray!6}  31 & male & 42 & 15.000 & yes\\
32 & female & 22 & 1.500 & no\\
\rowcolor{gray!6}  33 & male & 27 & 0.417 & no\\
34 & female & 42 & 15.000 & yes\\
\rowcolor{gray!6}  35 & male & 32 & 4.000 & yes\\
\addlinespace
36 & female & 22 & 1.500 & no\\
\rowcolor{gray!6}  37 & female & 42 & 15.000 & yes\\
38 & female & 22 & 4.000 & no\\
\rowcolor{gray!6}  39 & male & 22 & 1.500 & yes\\
40 & female & 22 & 0.750 & no\\
\addlinespace
\rowcolor{gray!6}  41 & male & 32 & 10.000 & yes\\
42 & male & 52 & 15.000 & yes\\
\rowcolor{gray!6}  43 & female & 22 & 0.417 & no\\
44 & female & 27 & 4.000 & yes\\
\rowcolor{gray!6}  45 & female & 32 & 7.000 & yes\\
\addlinespace
46 & male & 22 & 4.000 & no\\
\rowcolor{gray!6}  47 & female & 27 & 7.000 & yes\\
48 & female & 42 & 15.000 & yes\\
\rowcolor{gray!6}  49 & male & 27 & 1.500 & yes\\
50 & male & 42 & 15.000 & yes\\*
\end{longtabu}

%% ==================================================================

%%---------------- 致谢 ---------------
\cleardoublepage
\renewcommand{\chapterlabel}{\ackname} % 设置致谢参考文献的页眉

%%% 致谢

\begin{ack}
\begin{normalsize}
  \begin{spacing}{1.4}

在这里我要感谢...

我还要感谢...

\bigskip
\CJKfamily{kai}
\rightline{\large 姓名 \quad \ }
\rightline{2021年5月1日}

\end{spacing}
\end{normalsize}
\end{ack}




\end{document}
